\documentclass{dense_template}
\newcommand*{\name}{Felipe Alejandro Jiménez Castillo}
\newcommand*{\code}{215671386}
\newcommand*{\school}{Universidad de Guadalajara - CUCEI}
\newcommand*{\course}{Computación Tolerante a Fallas}
\newcommand*{\assignment}{Mastering Chaos - A Netflix Guide to Microservices}
\renewcommand{\contentsname}{Contenido}

\begin{document}
\maketitle
%%%%%%%%%%%%%%%%%%%%
\tableofcontents
\newpage
%%%%%%%%%%%%%%%%%%%%
\section{Introducción}
El documental \textit{"Mastering Chaos - A Netflix Guide to Microservices"} ofrece una perspectiva valiosa sobre los desafíos inherentes a la gestión de microservicios en entornos complejos como Netflix. El discurso destaca la importancia de la resiliencia, la escalabilidad y la adaptabilidad en la arquitectura de microservicios.

%%%%%%%%%%%%%%%%%%%%
\section{Reporte}
\subsection{Problemas con una arquitectura monolítica}
En el pasado, Netflix utilizaba una arquitectura monolítica, en la que todos los componentes de la aplicación se integraban en una única unidad. Esto presentaba una serie de desafíos, como la dificultad de escalar la aplicación, la dificultad de diagnosticar y resolver problemas, y la vulnerabilidad a los ataques.

\subsection{Arquitectura de microservicios}
Para superar estos desafíos, Netflix adoptó una arquitectura de microservicios. En este modelo, la aplicación se divide en una colección de servicios pequeños e independientes. Cada servicio es responsable de una función específica y se ejecuta en su propio entorno.

\subsection{Ventajas de la arquitectura de microservicios}
La arquitectura de microservicios ofrece una serie de ventajas, como:

\begin{itemize}
    \item \textbf{Resiliencia:} Los servicios individuales pueden fallar sin afectar a la disponibilidad general de la aplicación.
    \item \textbf{Escalabilidad:} Los servicios se pueden escalar de forma independiente para satisfacer las demandas de tráfico.
    \item \textbf{Adaptabilidad:} Los servicios se pueden modificar o reemplazar fácilmente para adaptarse a las necesidades cambiantes del negocio.
\end{itemize}

\subsection{Desafíos de la arquitectura de microservicios}
La arquitectura de microservicios también presenta algunos desafíos, como:

\begin{itemize}
    \item \textbf{Comunicación:} Los servicios deben comunicarse entre sí para intercambiar datos. Esto puede ser complejo y propenso a errores.
    \item \textbf{Gestión:} La gestión de una aplicación de microservicios puede ser compleja, ya que requiere la supervisión y el mantenimiento de múltiples servicios independientes.
\end{itemize}

\subsection{Lecciones aprendidas}
El documental destaca una serie de lecciones aprendidas por Netflix en su transición a la arquitectura de microservicios:
\begin{itemize}
    \item La transición a la arquitectura de microservicios debe ser gradual y cuidadosa.
    \item Es importante implementar prácticas sólidas de desarrollo y operación.
    \item Se deben utilizar herramientas y tecnologías adecuadas para la gestión de microservicios.
\end{itemize}

%%%%%%%%%%%%%%%%%%%%
\pagebreak
%%%%%%%%%%%%%%%%%%%%
\section{Conclusión}
La arquitectura de microservicios es una opción viable para aplicaciones complejas que requieren flexibilidad, escalabilidad y adaptabilidad. Sin embargo, es importante comprender los desafíos inherentes a este modelo antes de adoptarlo.

\begin{itemize}
    \item \textbf{Los microservicios están acoplados de forma débil.} Esto significa que están diseñados para ser independientes entre sí y no depender de otros servicios para su funcionamiento.
    \item \textbf{Los microservicios utilizan interfaces de programación de aplicaciones (API) para comunicarse entre sí.} Esto permite que los servicios se desarrollen y mantengan de forma independiente.
\end{itemize}

Para darnos una idea más clara podemos ejemplificar los beneficios de la arquitectura de microservicios. Por ejemplo, Netflix ha podido utilizar la arquitectura de microservicios para:
\begin{itemize}
    \item Aumentar la velocidad de entrega de nuevas características.
    \item Reducir el tiempo de inactividad de la aplicación.
    \item Mejorar la seguridad de la aplicación.
\end{itemize}
%%%%%%%%%%%%%%%%%%%%
\pagebreak
%%%%%%%%%%%%%%%%%%%%
\section{Bilbiografia}
\sloppy
\begin{enumerate}
    \item InfoQ [@infoq]. (2017, February 22). Mastering chaos - A Netflix guide to microservices. Youtube. https://www.youtube.com/watch?v=CZ3wIuvmHeM
    \item What is docker? (n.d.). Redhat.com. Retrieved October 24, 2023, from https://www.redhat.com/en/topics/containers/what-is-docker
\end{enumerate}
\end{document}

