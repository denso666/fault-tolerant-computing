\documentclass{dense_template}
\newcommand*{\name}{Felipe Alejandro Jiménez Castillo}
\newcommand*{\code}{215671386}
\newcommand*{\school}{Universidad de Guadalajara - CUCEI}
\newcommand*{\course}{Computación Tolerante a Fallas}
\newcommand*{\assignment}{Herramientas para el Manejo de Errores}
\renewcommand{\contentsname}{Contenido}

\begin{document}
\maketitle
%%%%%%%%%%%%%%%%%%%%
\tableofcontents
\newpage
%%%%%%%%%%%%%%%%%%%%
\section{Introducción}
El manejo de errores hace referencia a la respuesta y recuperación presentes en las condiciones de error que se encuentra un programa o software, en otras palabras es el proceso compuesto por la anticipación, detección y resolución de diferentes tipos de errores en una aplicación y/o programa. El manejo de errores nos ayuda a mantener el flujo de ejecución adecuado de un programa. En el día a día, muchas aplicaciones enfrentan numerosos retos de diseño cuando se consideran técnicas de manejo de errores, es por esto, que es todo una ciencia y arte conocer dichas técnicas para una implementación concisa.  
%%%%%%%%%%%%%%%%%%%%
\section{Desarrollo}
Una herramienta o software de manejo de errores, es aquel programa que nos ayuda, como desarrolladores, a identificar, rastrear y corregir el origen de un error. Estás herramientas pueden ser muy útiles para proyectos grandes o complejos, ya que nos ayudar a administrar y priorizar los errores, así como rastrear su origen y progreso. Con esto en mente una herramienta de manejo de errores debería tener al menos una de las siguientes características: Registro (ser capaz de identificar y denotar), Trazabilidad (ser capaz de rastrear), Priorizar (señalizar la gravedad) y Gestionar (asignar-seleccionar).
\begin{enumerate}
    \item \textbf{Jira}
    \begin{enumerate}
        \item Si bien \textit{Jira} es un software para la gestión de proyectos, en está se puede generar un estado de aplicación y realizar un seguimiento, asignación, priorización y gestión de los errores encontrados en nuestro software.
    \end{enumerate}
    \item \textbf{LSP}
    \begin{enumerate}
        \item \textit{LSP (Language Server Protocol)} es un servicio, creado por Microsoft, que agrega características como auto completado, ir a las definiciones, documentación, referencias y más, sobre el curso en el proceso de programación, funcionando como un protocolo de comunicación entre el IDE o editor y un servidor de lenguaje, lo que nos proporciona retroalimentación en tiempo real de errores, definiciones incorrectas, y más. 
    \end{enumerate}
    \item \textbf{Visual Studio Code}
    \begin{enumerate}
        \item Si bien \textit{Visual Studio Code} es un editor de código hecho y derecho, esté cuenta con una extensa cantidad de plugins que hacen del proceso de desarrollo un entorno más cómodo, con esté podemos tener herramientas de detección de errores pre-runtime, así como análisis léxico, sintáctico y semántica de la mayoría de lenguajes en el mundo, por lo que podríamos considerarlo como una herramienta muy potente.
    \end{enumerate}
\end{enumerate}
%%%%%%%%%%%%%%%%%%%%
\pagebreak
%%%%%%%%%%%%%%%%%%%%
\section{Conclusión}
Si bien el proceso de programación es un proceso lógico de mucha complejidad, el mundo moderno nos ha guiado al desarrollo de herramientas que nos ayudan a priorizar los retos lógicos y poder tener, de segunda mano, un asistente "inteligente" para solucionar errores, menores o mayores, que podamos dejar atrás o simplemente hayamos olvidado. Es aquí donde las herramientas de manejo de errores nos dan la mano para ser parte de nuestro proceso y habilitar una mejora a nuestros productos de software y, finalmente, mejorar la calidad del proceso de desarrollo.  
%%%%%%%%%%%%%%%%%%%%
\pagebreak
%%%%%%%%%%%%%%%%%%%%
\section{Bilbiografia}
\sloppy
\begin{enumerate}
    \item - Atlassian. (n.d.). Jira software - features. Atlassian. Retrieved August 27, 2023, from https://www.atlassian.com/software/jira/features
    \item Fault tolerance. (n.d.). Learning Center; Imperva Inc. Retrieved August 15, 2023, from https://www.imperva.com/learn/availability/fault-tolerance/
    \item Extensions, L. M. A. (n.d.). Visual Studio Code - code editing. Redefined. Visualstudio.com. Retrieved August 27, 2023, from https://code.visualstudio.com/
    \item Gunnell, M., Haqshanas, R., \& Cooling, S. (2017, May 1). Error Handling. Techopedia. https://www.techopedia.com/definition/16626/error-handling
    \item Official page for Language Server Protocol. (n.d.). Github.Io. Retrieved August 27, 2023, from https://microsoft.github.io/language-server-protocol/
\end{enumerate}
\end{document}




